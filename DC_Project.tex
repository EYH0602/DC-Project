
\documentclass[12pt]{article}

%
%Margin - 1 inch on all sides
%
\usepackage[letterpaper]{geometry}
\usepackage{times}
\geometry{top=1.0in, bottom=1.0in, left=1.0in, right=1.0in}

%
%Doublespacing
%
\usepackage{setspace}
\doublespacing

%
%Rotating tables (e.g. sideways when too long)
%
\usepackage{rotating}

%
% Indent the first paragraph after section title
%
\usepackage{indentfirst}

% use roman numerals for section and subsection
\renewcommand{\thesection}{\Roman{section}} 
\renewcommand{\thesubsection}{\thesection.\Roman{subsection}}

%
%allow counting superscript
%
\usepackage[super]{nth}

% set size of section title
\usepackage{titlesec}
\titleformat*{\section}{\normalfont\bfseries}
\titleformat*{\subsection}{\normalfont\bfseries}

%
%Fancy-header package to modify header/page numbering (insert last name)
%
\usepackage{fancyhdr}
\pagestyle{fancy}
\lhead{} 
\chead{} 
\rhead{He \thepage} 
\lfoot{} 
\cfoot{} 
\rfoot{} 
\renewcommand{\headrulewidth}{0pt} 
\renewcommand{\footrulewidth}{0pt} 
%To make sure we actually have header 0.5in away from top edge
%12pt is one-sixth of an inch. Subtract this from 0.5in to get headsep value
\setlength\headsep{0.333in}


%
%Works cited environment
%(to start, use \begin{workscited...}, each entry preceded by \bibent)
% - from Ryan Alcock's MLA style file
%
\newcommand{\bibent}{\noindent \hangindent 40pt}
\newenvironment{workscited}{\newpage \begin{center} Works Cited \end{center}}{\newpage }


%
%Begin document
%
\begin{document}
\begin{flushleft}

%%%%First page name, class, etc
Ethan He \\
Professor Miller \\
UWP 001 \\
August 22 2020 \\

% title
\begin{center}
    \textbf{Gaining Voice: Transformations of the Undocumented Youth Movement in CIJYA}
\end{center}

%%%%Changes paragraph indentation to 0.5in
\setlength{\parindent}{0.5in}
%%%%Begin body of paper here

\section*{Cover Memo}

\noindent
Dear Peer Responder,

% one paragraph describing what you think the strengths and weaknesses of the draft are
I think my draft's strength is its clear thesis and structure.
My thesis, stated in the Abstract section, is focused and specific.
About the structure, there are three aspects of the function of storytelling.
They build up to support the main claim of my paper.
The current weakness of this draft is that the second and third section is leaking detail and description.
Some additional sources to support the third point is also needed.

% one paragraph with any questions or concerns you have for your peer responders
I have two questions about this draft.
First, I would like to know if my thesis is strong enough.
Second, 

\noindent
Sincerely, \\
\noindent
Ethan He

\section{Abstract}

In the past decades, more undocumented youth have come of the shadows to struggle against the criminalization and discrimination they encountered and fight for their rights in community, society, and the United States. 
Among the surging youth-led alliances, California Immigrant Youth Justice Alliance (CIYJA) is one of the most influential organizations that focus on placing undocumented youth in advocacy and policy delegations. 
By strategically employing empowering practices, CIYJA launches a transformational movement that goes beyond narrowly defined legalization and attempts to gain radical egalitarian citizenship.
This paper focuses on their strategies of storytelling and organizing diverse communities, analyzes and argues how these practices give voice to community members, expand the Dreamer narrative, and mobilize undocumented youth as well as the public.

\section{The Power of Storytelling}

Storytelling is one of the most predominant strategies the CIYJA employs as an identity-(re)constructing, community-building, and appeal-making practice in the undocumented youth movement.
% identity-(re)constructing: a new understanding to my identity after some experience.
Berger states that human life is narratively rooted. % source
Riessman stresses the ability of narratives to ``do political work'' (8) in constructing norms, identities, and ideologies. % source
The CIYJA’s members tell their stories in different forms and these narratives are employed strategically to achieve diverging goals based on their messages, contexts, and audience.
In this section, I select and analyze three stories from different platforms of CIYJA to showcase how they serve effectively the common goals of the organization.

\subsection*{Storytelling as a practice to (re)construct and negotiate identity}

The first story is told by Mariela Mendez, a Cultivator of Change with CIYJA, in an article under the name of ``Our Voice Will Not be Drowned'' on CIYJA's website. In her story, Mariela shares her experience of coming out shadows painstakingly yet bravely:

\begin{quotation}
    \noindent
    My family's time here has been a story of shadows. … [T]hese shadows have silenced me.
    For some odd reason, I felt that it was necessary to remain quiet and unseen from protests that involved controversy. 
    I didn’t want to implicate or worry about my family.
    I still recall living in the shadows with fear, and uncertainty in my future.
    Despite knowing that the rhetoric and terms used to identify the undocumented community led to the unethical and inhumane treatment I experienced, I was okay with living in the shadows.

    \noindent $\ldots$
    
    \noindent
    The urgency to feel that it is necessary to remain quiet and to not have attention drawn to us for sake of our family is a feeling all too common amongst the undocumented community.
    However, this damning silence is something that I was able to break away from.
\end{quotation} % source-checked

As most undocumented youth, Mariela initially accepted the social identity assigned to her and her community, and ``was OK living in the shadows'' until an incident at the TRUTH Forum in August 2018, where she witnessed paid agitator yelling out racist and homophobic slurs at them, a group of local undocumented youth helping with organizing the forum.
By recalling and reflecting the experience, Mariela gained determination and bravery to rise up from the shadows regardless of her legal status.
Suganami suggests that narratives are essential to community building and (re)constructing a common identity. % source
In this case, Mariela not only (re)constructs and negotiates her collective identity within the community, but more importantly, she obtains subjectivity, a sense of agency, to reject and resist the assigned identity and make assertive claims, as she narrates at the end of her story:

\begin{quotation}
    \noindent
    I know that in order to start living out of shadows, I must not be afraid to stand up for what I believe in, and I must learn to be fearless.
    It is of vital importance for all who have immigrated to the U.S. with or without a current legal status to come out of the shadows in order to raise awareness and publicly advocate for themselves.
\end{quotation} % source-checked

\subsection*{Storytelling as a practice to build diverse communities}

Despite the initially good meaning of “Dreamer”, this term has been rejected by more and more undocumented youth and movement organizers for its labeling implication and exclusive effect.
For example, Edna, the CIYJA organizer suggested, 
    ``It was kinda helpful, but at the same time I think we really thought about the consequences of how it would create the two categories: `good immigrant' versus `bad immigrant''' (Schwiertz 615). % a reference in my reference, how to cite? - checked
Similarly, under the title of ``Take It from the Central Valley: You're Using the Wrong Narrative'', Brisa Cruz shares her disappointment when reading such an article as ``Want to Send Dreamers back to Mexico?'':

\begin{quotation}
    \noindent
    The story only associates the current immigration climate with the so called ``Dreamers,'' a label that I strongly dislike and have never identified with.
    I am a DACA recipient myself and understand the privilege I have; however, a social security number and an Employment Authorization Card doesn’t mean I have stopped fighting for the human dignity and liberation of my community--the same community that continues to be criminalized and tokenized by those in power.
\end{quotation} % source-checked

In an attempt to organize diverse communities, the CIYJA uses individual storytelling to create a broader, more inclusive narrative, which connects different fights against the anti-migration hegemony.
In a video clip posted by Daniel Alvarenga and retweeted by the CIYJA, several Cameroonian immigrants detained at Pine Prairie ICE shared their stories of being caught and locked up in solitary confinement.
In the campaign of ``Free Them All'', Ciyia Valeria, on her Instagram post, read a letter from someone detained at Mesa Verde Detention Facility, telling their unknown stories in the facility.
These stories display the diversity of the undocumented community and build a more inclusive movement.

\subsection*{Storytelling as a practice to spread claims}

Besides displaying their emotions, concerns as the subject of brutal policing, incarceration and deportation, all these self- and other-told stories also serve as ``significant part of extra-movement communication with the media, public and politicians'' (Swerts 355). % source-checked
In the petition of ``Help Erika Return to Her Family'' made to Congress members, the organizers told the story of how Erica and her family struggled for her legal citizenship. % in-text citation needed
Having lived in the Central Valley since 1999, Erica was rejected in a visa application in 2014 due to a previous unauthorized re-entry into the U.S when she was a minor back in 2006.
This type of storytelling ``provides politicians with the moral and emotional resources necessary to legitimate their support'' (Swerts 356).
By telling stories to urge Congress members to help Erika obtain discretion in her case, the CIYJA makes the appeal that ``We need more politicians to take a public stance against the rogue undermining of the current administration, for the safety of all targeted communities''. % https://salsa4.salsalabs.com/o/51625/p/dia/action4/common/public/?action_KEY=24979 , need to cite

\section{Conclusion}

Storytelling is powerful assets for marginalized communities and individuals.
By giving their voices, reconstructing their identities and subjectivity, and justifying their claims, these stories connect not only the community members but also the human history.
As Emanuel claims, ``Undocumented immigrants are the backbone of the entire southwest region of the United States, and we are not going anywhere.'' % source

%%%%Works cited
\begin{workscited}

\bibent
Cruz, Brisa. ``Take It from the Central Valley: You're Using the Wrong Narrative''. \textit{CIYJA}. https://ciyja.org/take-it-from-the-central-valley/

\bibent
Mendez, Mariela. ``Our Voices Will Not Be Drowned''. \textit{CIYJA}. https://ciyja.org/our-voices-will-not-be-drowned/

\bibent
Schwiertz, Helge. ``Transformations of the Undocumented youth movement and radical egalitarian citizenship.'' \textit{Citizenship Studies}, vol. 20, no. 5, 2016, pp.610-628.

\bibent
Swerts, Thomas. “Gaining a voice: Storytelling and undocumented youth activism in Chicago.” Mobilization: An International Quarterly, vol. 20, no. 3, 2015, pp. 345-360.


% Berger states that human life is narratively rooted. 
% Riessman stresses the ability of narratives to ``do political work'' (8) in constructing norms, identities, and ideologies.
% Suganami suggests that narratives are essential to community building and (re)constructing a common identity.
% As Emanuel claims, ``Undocumented immigrants are the backbone of the entire southwest region of the United States, and we are not going anywhere.'' % source

\end{workscited}

\end{flushleft}
\end{document}
