
\documentclass[12pt]{article}

%
%Margin - 1 inch on all sides
%
\usepackage[letterpaper]{geometry}
\usepackage{times}
\geometry{top=1.0in, bottom=1.0in, left=1.0in, right=1.0in}

%
%Doublespacing
%
\usepackage{setspace}
\doublespacing

%
%Rotating tables (e.g. sideways when too long)
%
\usepackage{rotating}

%
% Indent the first paragraph after section title
%
\usepackage{indentfirst}

% use roman numerals for section and subsection
\renewcommand{\thesection}{\Roman{section}} 
\renewcommand{\thesubsection}{\thesection.\Roman{subsection}}

%
%allow counting superscript
%
\usepackage[super]{nth}

% set size of section title
\usepackage{titlesec}
\titleformat*{\section}{\normalfont\bfseries}

%
%Fancy-header package to modify header/page numbering (insert last name)
%
\usepackage{fancyhdr}
\pagestyle{fancy}
\lhead{} 
\chead{} 
\rhead{He \thepage} 
\lfoot{} 
\cfoot{} 
\rfoot{} 
\renewcommand{\headrulewidth}{0pt} 
\renewcommand{\footrulewidth}{0pt} 
%To make sure we actually have header 0.5in away from top edge
%12pt is one-sixth of an inch. Subtract this from 0.5in to get headsep value
\setlength\headsep{0.333in}

%
%Works cited environment
%(to start, use \begin{workscited...}, each entry preceded by \bibent)
% - from Ryan Alcock's MLA style file
%
\newcommand{\bibent}{\noindent \hangindent 40pt}
\newenvironment{workscited}{\newpage \begin{center} Works Cited \end{center}}{\newpage }


%
%Begin document
%
\begin{document}
\begin{flushleft}

%%%%First page name, class, etc
Ethan He \\
Professor Miller \\
UWP 001 \\
August 22 2020 \\

% title
\begin{center}
    \textbf{Gaining Voice: Transformations of the Undocumented Youth Movement in CIJYA}
\end{center}

%%%%Changes paragraph indentation to 0.5in
\setlength{\parindent}{0.5in}
%%%%Begin body of paper here

\section*{Cover Memo}

\noindent
Dear Peer Responder,

one paragraph describing what you think the strengths and weaknesses of the draft are

one paragraph with any questions or concerns you have for your peer responders

\noindent
Sincerely, \\
\noindent
Ethan He

\section{Thesis}

In the past decades, more undocumented youth have come of the shadows to struggle against the criminalization and discrimination they encountered and fight for their rights in community, society, and the United States. 
Among the surging youth-led alliances, California Immigrant Youth Justice Alliance (CIYJA) is one of the most influential organizations that focus on placing undocumented youth in advocacy and policy delegations. 
By strategically employing empowering practices, CIYJA launches a transformational movement that goes beyond narrowly defined legalization and attempts to gain radical egalitarian citizenship.
This paper focuses on their strategies of storytelling and organizing diverse communities, analyzes and argues how these practices give voice to community members, expand the Dreamer narrative, and mobilize undocumented youth as well as the public.

\section{The Power of Storytelling}




%%%%Works cited
\begin{workscited}

\bibent
Negr\'{o}n-Gonzales, Genevieve. ``Navigating `Illegality': Undocumented Youth \& Oppositional Consciousness.'' \textit{Children and Youth Services Review}, vol. 35, no. 8, 2013, pp. 1284-1290

\bibent
Spade, Dean. ``Solidarity Not Charity: Mutual Aid for Mobilization and Survival.'' \textit{Social Text}, 38.1, 2020, pp. 131-151


\end{workscited}

\end{flushleft}
\end{document}
