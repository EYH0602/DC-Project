
\documentclass[12pt]{article}

%
%Margin - 1 inch on all sides
%
\usepackage[letterpaper]{geometry}
\usepackage{times}
\geometry{top=1.0in, bottom=1.0in, left=1.0in, right=1.0in}

%
%Doublespacing
%
\usepackage{setspace}
\doublespacing

%
%Rotating tables (e.g. sideways when too long)
%
\usepackage{rotating}

%
% Indent the first paragraph after section title
%
\usepackage{indentfirst}

% use roman numerals for section and subsection
\renewcommand{\thesection}{\Roman{section}} 
\renewcommand{\thesubsection}{\thesection.\Roman{subsection}}

%
%allow counting superscript
%
\usepackage[super]{nth}

% set size of section title
\usepackage{titlesec}
\titleformat*{\section}{\normalfont\bfseries}

%
%Fancy-header package to modify header/page numbering (insert last name)
%
\usepackage{fancyhdr}
\pagestyle{fancy}
\lhead{} 
\chead{} 
\rhead{He \thepage} 
\lfoot{} 
\cfoot{} 
\rfoot{} 
\renewcommand{\headrulewidth}{0pt} 
\renewcommand{\footrulewidth}{0pt} 
%To make sure we actually have header 0.5in away from top edge
%12pt is one-sixth of an inch. Subtract this from 0.5in to get headsep value
\setlength\headsep{0.333in}

%
%Works cited environment
%(to start, use \begin{workscited...}, each entry preceded by \bibent)
% - from Ryan Alcock's MLA style file
%
\newcommand{\bibent}{\noindent \hangindent 40pt}
\newenvironment{workscited}{\newpage \begin{center} Works Cited \end{center}}{\newpage }


%
%Begin document
%
\begin{document}
\begin{flushleft}

%%%%First page name, class, etc
Ethan He \\
Professor Miller \\
UWP 001 \\
August 22 2020 \\

% title
\begin{center}
    \textbf{Where Everything Started: the Subjective Identity Belongingness of Undocumented Youth}
\end{center}

%%%%Changes paragraph indentation to 0.5in
\setlength{\parindent}{0.5in}
%%%%Begin body of paper here

\section*{Cover Memo}

\noindent
Dear Peer Responder,

one paragraph describing what you think the strengths and weaknesses of the draft are

one paragraph with any questions or concerns you have for your peer responders

\noindent
Sincerely, \\
\noindent
Ethan He

\section{Background}

The existence of undocumented immigrants had been around for decades, taking over 3.2\% of the total population in the United States in 2017 (Krogstad et al.). 
The topic got itself to the public along with the introduction to the Development, Relief, and Education for Alien Minors Act (commonly known as the DREAM Act) 20 years ago. This act opened a path for undocumented youth to become an American citizen (Hager). 
However, the DREAM Act was unreasonable in certain aspects of this broad community. 
Once the undocumented youths committed any crime, important or not, they lose the right to obtain citizenship. 
Along with the absurd points in the DREAM Act, social injustices and racism were also factors that cause the Undocumented Youth movement. The government sometimes creates a reason to deport undocumented immigrants (Carrasco et al. 284). 

% more background needed

\section{Argument}

The collective identity is the most important weapon for the DREAMers to fight for their rights in the United States. 
Although any individual is weak, together they can make changes. 
But how did the DREAMers come together and share their force at the beginning? 
The major causes of the Undocumented Youth movement are obvious: social injustices, unfair treatments, and discrimination. 
Beyond the outside rationales, Fiorito claims that subjectivities also play an important role in the social movement (345). 
In her opinion, the subjectivities of the undocumented immigrants are the result of their similar previous experience and shared affective (Fiorito 346). She further refers to this subjectivities as a "sense of belonging" (Fiorito 347). 
To extend on Fiorito's argument of subjectivity in the Undocumented Youth movement, the connotation of such social movements is the subjective identity belongingness that contributes to their collective force.


\section{Analysis}

\section{Conclusion}

%%%%Works cited
\begin{workscited}

\bibent
Negr\'{o}n-Gonzales, Genevieve. ``Navigating `Illegality': Undocumented Youth \& Oppositional Consciousness.'' \textit{Children and Youth Services Review}, vol. 35, no. 8, 2013, pp. 1284-1290

\bibent
Spade, Dean. ``Solidarity Not Charity: Mutual Aid for Mobilization and Survival.'' \textit{Social Text}, 38.1, 2020, pp. 131-151


\end{workscited}

\end{flushleft}
\end{document}
