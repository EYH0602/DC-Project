
\documentclass[12pt]{article}

%
%Margin - 1 inch on all sides
%
\usepackage[letterpaper]{geometry}
\usepackage{times}
\geometry{top=1.0in, bottom=1.0in, left=1.0in, right=1.0in}

%
%Doublespacing
%
\usepackage{setspace}
\doublespacing

%
%Rotating tables (e.g. sideways when too long)
%
\usepackage{rotating}

%
% Indent the first paragraph after section title
%
\usepackage{indentfirst}

% use roman numerals for section and subsection
\renewcommand{\thesection}{\Roman{section}} 
\renewcommand{\thesubsection}{\thesection.\Roman{subsection}}

%
%allow counting superscript
%
\usepackage[super]{nth}

% set size of section title
\usepackage{titlesec}
\titleformat*{\section}{\normalfont\bfseries}
\titleformat*{\subsection}{\normalfont\bfseries}

%
%Fancy-header package to modify header/page numbering (insert last name)
%
\usepackage{fancyhdr}
\pagestyle{fancy}
\lhead{} 
\chead{} 
\rhead{He \thepage} 
\lfoot{} 
\cfoot{} 
\rfoot{} 
\renewcommand{\headrulewidth}{0pt} 
\renewcommand{\footrulewidth}{0pt} 
%To make sure we actually have header 0.5in away from top edge
%12pt is one-sixth of an inch. Subtract this from 0.5in to get headsep value
\setlength\headsep{0.333in}


%
%Works cited environment
%(to start, use \begin{workscited...}, each entry preceded by \bibent)
% - from Ryan Alcock's MLA style file
%
\newcommand{\bibent}{\noindent \hangindent 40pt}
\newenvironment{workscited}{\newpage \begin{center} Works Cited \end{center}}{\newpage }


%
%Begin document
%
\begin{document}
\begin{flushleft}

%%%%First page name, class, etc
Ethan He \\
Professor Miller \\
UWP 001 \\
August 31 2020 \\

% title
\begin{center}
    \textbf{Gaining Voice: Transformations of the Undocumented Youth Movement in CIJYA}
\end{center}

%%%%Changes paragraph indentation to 0.5in
\setlength{\parindent}{0.5in}
%%%%Begin body of paper here

\section*{Cover Memo}

\noindent
Dear Instructor Miller,

% pros and cons
The strengths of this paper, according to the peer reviews as well, lie in its specific, focused central claim and clear structure. 
Accurate, relevant secondary sources are also used properly to provide solid evidence. 
By analyzing the power of storytelling from three aspects, the paper offers well-organized sub-claims which build up to support the main thesis.

% question and concerns
I got the inspiration of analyzing storytelling as a rhetorical strategy from one of the previous course assignments, an interview-based bibliography of a family member. 
When I interviewed my grandmother in a formal way, I found she told some stories differently from what she did before. 
Storytelling worked in an interesting way to construct and reconstruct meanings. 
Before working on this paper, I referred to some literature on storytelling in prison abolition movement and found surprisingly storytelling has been used as a political tool to promote social change. 
So, it is not original to analyze the power of storytelling in prison abolition movement, but I learned a lot from reading scholarship on storytelling, based on which I composed my first formal research paper. 

% summarizing peer feedback
Besides the lack of originality in structure, I have also concern about the analysis part in the paper. 
According to the one of the peer reviews, it would be better if more analysis is added before and after the citation, which I managed to do in the second draft. 
Another peer review points out my conclusion is a little weak, which I haven’t revised dramatically in this draft because of time and length constraint, but I will revise it in the final draft. 
Other revisions I have made after reading peer reviews also include
    1) to balance the weight of citations and my own voice; 
    2) to add better transitions within paragraphs, especially after citations. 
Finally, I am not sure about the format of long quotes. 
I adopted the formatting instructed on the OWL website, with $\frac{1}{2}$ inch indent from the left margin, while some websites state it should be 1 inch. 

\noindent
Sincerely, \\
\noindent
Ethan He

\section{Introduction}

In the past decades, more undocumented youth have come of the shadows to struggle against the criminalization and discrimination they encountered and fight for their rights in community, society, and the United States. 
Among the surging youth-led alliances, California Immigrant Youth Justice Alliance (CIYJA) is one of the most influential organizations that focus on placing undocumented youth in advocacy and policy delegations. 
By strategically employing empowering practices, CIYJA launches a transformational movement that goes beyond narrowly defined legalization and attempts to gain radical egalitarian citizenship.
This paper focuses on their strategies of storytelling and organizing diverse communities, analyzes and argues how these practices give voice to community members, expand the Dreamer narrative, and mobilize undocumented youth as well as the public.

\section{The Power of Storytelling}

Jerome Bruner has argued that one of the important ways people understand their world is through storytelling, 
in which people express their wants, needs, and goals. % source-checked
Davis points out that ``social movements are dominated by stories and storytelling, and narratives goes to the heart of the very cultural and ideational processes, including public discourse, movement culture, $\ldots$ and collective identity (4)''. % source-checked
Riessman also stresses the ability of narratives to ``do political work'' (8) in constructing norms, identities and ideologies. % source-checked
Although playing a critical role in social movements, storytelling and its transformational power has been relatively neglected in the research on undocumented youth community. 
This paper examines how the CIYJA's members tell their stories in different forms and how these narratives are employed strategically to achieve diverging goals. 
% thesis
Three stories from different platforms of CIYJA are select and analyzed in the following sections to showcase their efficiency to serve the common goals of the organization.

\subsection*{Storytelling (re)constructs and negotiates identity}

The first story is told by Mariela Mendez, a Cultivator of Change with CIYJA, in an article under the name of ``Our Voice Will Not be Drowned'' on CIYJA's website. In her story, Mariela shares her experience of coming out shadows painstakingly yet bravely:

\begin{quotation}
    \noindent
    My family's time here has been a story of shadows. $\ldots$ [T]hese shadows have silenced me.
    For some odd reason, I felt that it was necessary to remain quiet and unseen from protests that involved controversy. 
    I didn't want to implicate or worry about my family.
    I still recall living in the shadows with fear, and uncertainty in my future.
    Despite knowing that the rhetoric and terms used to identify the undocumented community led to the unethical and inhumane treatment I experienced, I was okay with living in the shadows.

    \noindent $\ldots$
    
    \noindent
    The urgency to feel that it is necessary to remain quiet and to not have attention drawn to us for sake of our family is a feeling all too common amongst the undocumented community.
    However, this damning silence is something that I was able to break away from.
\end{quotation} % source-checked

As most undocumented youth, Mariela initially accepted the social identity assigned to her and her community, and ``was OK living in the shadows'' until an incident at the TRUTH Forum in August 2018, where she witnessed paid agitator yelling out racist and homophobic slurs at them, a group of local undocumented youth helping with organizing the forum.
By recalling and reflecting the experience, Mariela gained determination and bravery to rise up from the shadows regardless of her legal status.
Suganami suggests that narratives are essential to community building and (re)constructing a common identity. % source-checked
In this case, Mariela not only (re)constructs and negotiates her collective identity within the community, but more importantly, she obtains subjectivity, a sense of agency, to reject and resist the assigned identity and make assertive claims, as she narrates at the end of her story:

\begin{quotation}
    \noindent
    I know that in order to start living out of shadows, I must not be afraid to stand up for what I believe in, and I must learn to be fearless.
    It is of vital importance for all who have immigrated to the U.S. with or without a current legal status to come out of the shadows in order to raise awareness and publicly advocate for themselves.
\end{quotation} % source-checked

This story shared on the CIYJA's website with most of its audiences as undocumented youth and supporters works effectively to gain voice to used-to-be silent groups. 
By giving themselves new, positive identities, undocumented youth are empowered to challenge anti-immigrant rhetoric and advocate for themselves.

\subsection*{Storytelling builds diverse communities}

Storytelling, with it emotional, cultural mechanism, can be used as a powerful tool to build more diverse communities (Swerts). % source-checked
One story shared on the CIYJA's website, which challenges the narrative of ``Dreamer'', serves such a goal as promoting a more inclusive, diverse community by directly opposing the criminalization of undocumented youth. % sources needed
Since the DREAM Act passed in 2019 after years of debate, the term ``Dreamer'' has been increasingly used by researchers and community members as a collective identity of undocumented youth. 
However, despite its function in generating and maintaining movement and promoting collective action (Fiorito), this term has been rejected by more undocumented youth and movement organizers for its labeling implication and exclusive effect. 
For example, Edna, the CIYJA organizer suggests, ``[The term] was kinda helpful, but at the same time I think we really thought about the consequences of how it would create the two categories: ‘good immigrant' versus ‘bad immigrant''' (qtd. in  Schwiertz 615). % source-checked
Similarly, under the title of ``\textit{Take It from the Central Valley: You're Using the Wrong Narrative}'', Brisa Cruz, the Central California Regional Organizer with CIYJA, shares her disappointment when reading such a well-meaning article as ``\textit{Want to Send Dreamers back to Mexico? If you met one, you'd probably change your mind}'' on the news website of The Fresno Bee:

\begin{quotation}
    \noindent
    The story only associates the current immigration climate with the so called ``Dreamers,'' a label that I strongly dislike and have never identified with.
    I am a DACA recipient myself and understand the privilege I have; however, a social security number and an Employment Authorization Card doesn't mean I have stopped fighting for the human dignity and liberation of my community--the same community that continues to be criminalized and tokenized by those in power.
\end{quotation} % source-checked

In her narrative, Brisa showed her strong emotions about being thrust upon the rhetoric of ``Dreamer''. 
Her advocate in the later article to find ``creative, inclusive alternatives that involves everyone in our diverse community'', demonstrate how storytelling, as relates to emotions, can raise self- and other's awareness of destructing the narrowly defined narrative. 
In an attempt to organize diverse communities, the CIYJA uses individual storytelling to create a broader, more inclusive narrative, which connects different fights against the anti-migration hegemony.
In a video clip posted by Daniel Alvarenga and retweeted by the CIYJA, several Cameroonian immigrants detained at Pine Prairie ICE shared their stories of being caught and locked up in solitary confinement. % not sure if I need to cite this tweet
In the campaign of ``Free Them All'', Ciyia Valeria, on her Instagram post, read a letter from someone detained at Mesa Verde Detention Facility, telling their unknown stories in the facility. % not source if I need to cite instagram
These stories display the diversity of the undocumented community and build a more inclusive movement.

\subsection*{Storytelling makes claims}

Besides displaying their emotions, concerns as the subject of brutal policing, incarceration and deportation, all these self- and other-told stories also serve as ``significant part of extra-movement communication with the media, public and politicians'' (Swerts 355). % source-checked
When being purposely used in different contexts, with different audience, the stories provide the moral grounds to gain public support.
In the petition of ``Help Erika Return to Her Family'' made to Congress members, the organizers told the story of how Erica and her family struggled for her legal citizenship. 
Having lived in the Central Valley since 1999, Erica was rejected in a visa application in 2014 due to a previous unauthorized re-entry into the U.S when she was a minor back in 2006.
This type of storytelling ``provides politicians with the moral and emotional resources necessary to legitimate their support'' (Swerts 356).
By telling stories to urge Congress members to help Erika obtain discretion in her case, the CIYJA makes the appeal that ``We need more politicians to take a public stance against the rogue undermining of the current administration, for the safety of all targeted communities''. 

\section{Conclusion}

Storytelling is powerful assets for marginalized communities and individuals.
By giving their voices, reconstructing their identities and subjectivity, and justifying their claims, these stories connect not only the community members but also the human history.
As Emanuel claims, ``Undocumented immigrants are the backbone of the entire southwest region of the United States, and we are not going anywhere. (qtd. in Servin)'' % source-checked

%%%%Works cited
\begin{workscited}

\bibent
Bruner, Jerome S. \textit{Acts of Meaning}. Harvard University Press, 1990.

\bibent
Cruz, Brisa. ``Take It from the Central Valley: You're Using the Wrong Narrative''. \textit{CIYJA}. https://ciyja.org/take-it-from-the-central-valley/

\bibent
Davis, Joseph E. \textit{Stories of Change: Narrative and Social Movements}. State University of New York Press, 2002.

\bibent
Mendez, Mariela. ``Our Voices Will Not Be Drowned''. \textit{CIYJA}. https://ciyja.org/our-voices-will-not-be-drowned/

\bibent
Riessman, Catherine Kohler. \textit{Narrative Methods for the Human Sciences}. Sage Publications, 2007.

\bibent
Schwiertz, Helge. ``Transformations of the Undocumented youth movement and radical egalitarian citizenship.'' \textit{Citizenship Studies}, vol. 20, no. 5, 2016, pp.610-628.

\bibent
Servin, John. ``Press Release: Kern Country `Coming out of the Shadows'''. \textit{CIYJA}. https://ciyja.org/kern-county-coming-out-of-the-shadows-march-23/

\bibent
Swerts, Thomas. ``Gaining a voice: Storytelling and undocumented youth activism in Chicago.'' Mobilization: An International Quarterly, vol. 20, no. 3, 2015, pp. 345-360.

\bibent
``Take Action: Help Erika Return To Her Family''. \textit{CIYJA}. https://ciyja.org/petitions/
\end{workscited}

\end{flushleft}
\end{document}
